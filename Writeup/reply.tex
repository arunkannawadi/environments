\documentclass[english]{letter}
\usepackage[dvipsnames]{color}
\usepackage{framed}
\definecolor{shadecolor}{named}{GreenYellow}
\usepackage{babel}


\newcommand{\rachel}[1]{\textrm{\textcolor{green}{Rachel: #1}}}
\newcommand{\arun}[1]{\textrm{\textcolor{red}{Arun: #1}}}
\newcommand{\claire}[1]{\textrm{\textcolor{blue}{Claire: #1}}}

\begin{document}


\date{\today}

\rachel{A basic comment about format: the MNRAS system will force you to upload a separate reply to
  the editor and to the authors.  So having it together in a single compiled PDF is not convenient.
  You should split it up into two files.  Also I'm not sure the system will let you enter your
  comments to the editor as a PDF, it might have to be text.}

\begin{letter}{Ms. Anna Evripidou\\ Assistant Editor\\ Monthly Notices of the 
Royal Astronomical Society}

\opening{Ms. Evripidou:}

We are very grateful to the referee for the careful reading of the manuscript.
We resubmit a modified version taking care of the comments and criticisms of the referee, plus
additional minor changes that are also explained in the report to the referee.
 
Please find below a point-by-point response to the referee report.

\closing{Sincerely,}
Arun Kannawadi, Rachel Mandelbaum, and Claire Lackner.
\end{letter}
\vspace{20pt}
{\bf Reply to the Referee}

We thank the referee for carefully reading our manuscript and for the acknowledgment that our work is of interest
and suitable for MNRAS. We are grateful for the pointers for improving the presentation and
explanation of our work. 
Below we address the specific issues raised, and note that you can find all new or edited text in
the paper in red.

\begin{shaded}
It is not clear in b/w which histogram is which in the top panels of figs. 6 and beyond. One appears to be lighter than the other, but it isn't stated in the figures which linetype corresponds to which samples.
\end{shaded}
\noindent
We agree with the referee that in black \& white print, there is potential for the reader to get confused in Figs. 6-8.
We have now rectified it by having different linestyles whenever more than one curves are present in the panel. In fact, we have extended this to Fig. 1 to Fig. 5 as well. We believe that the legends in the plots are self-explanatory after changing the linestyles.

\begin{shaded}
The number of galaxies in OD or UD bins is often quite different (factor of 2-4?). It would be useful to comment on whether this difference can account for some of the differences in mean quantities plotted. I think the effect due to differences in the number of galaxies should be small for rms(e), but a specific comment on the relative effect would be useful. A straightforward way to verify this is to repeat the calculation on a selection of the same number of galaxies from both populations.
\end{shaded}
\noindent 
The number of
galaxies in overdense bins is higher by roughly a factor of 2 in all the plots we have made. Table 1
is now updated to reflect that we plot $z=0.65-0.7$ and $z=0.7-0.75$ as two separate overdense bins.
(The previous version, which combined them, erroneously implied the factor of 4 mentioned in the
referee's comments.)

We performed the analysis again using a random subset of 50 per cent of the galaxies in overdense
regions, keeping the underdense bins intact. The uncertainities in the overdense bins increase due
to smaller sample size. However, the difference is not significant enough to alter our conclusions
based on the $p$-values from the KS and AD tests. We have noted this in Section 4.2.

\begin{shaded}
Is there an expectation that the 'neutral' bins should fall between OD and UD bins in all or some of the scatter plots? That doesn't seem to be the case for all. The trend when considering the neutral points confuses the conclusion of an effect due to structure in some quantities/plots. These points should be addressed explicitly. For example, presumably $f_{bulge}$ in the neutral bins should fall between OD and UD bins, given the argument at the end of page 13. It is hard to see from Fig. 15 that this is the case, and thus whether the difference between OD and UD bins is significant at all, which would not support the suggested mitigation method (b) discussed on page 13, which fig 15 is meant to test.
\end{shaded}

Not really. It is difficult to make any definitive statements about the behavior in neutral bins. As we have verified and pointed out in the paper after revision,
the neutral bins in our sample consists of overdensities and underdensities making it look like neutral in the number count. However, there are more galaxies in the OD regions than in the UD regions thus making the quantities measured more overdense-like. In almost all of the scatterplots (Figs.9-11), the green points (neutral)tend to line up with the red points (OD). 
\end{document}
